% Options for packages loaded elsewhere
\PassOptionsToPackage{unicode}{hyperref}
\PassOptionsToPackage{hyphens}{url}
%
\documentclass[
]{article}
\usepackage{amsmath,amssymb}
\usepackage{lmodern}
\usepackage{ifxetex,ifluatex}
\ifnum 0\ifxetex 1\fi\ifluatex 1\fi=0 % if pdftex
  \usepackage[T1]{fontenc}
  \usepackage[utf8]{inputenc}
  \usepackage{textcomp} % provide euro and other symbols
\else % if luatex or xetex
  \usepackage{unicode-math}
  \defaultfontfeatures{Scale=MatchLowercase}
  \defaultfontfeatures[\rmfamily]{Ligatures=TeX,Scale=1}
\fi
% Use upquote if available, for straight quotes in verbatim environments
\IfFileExists{upquote.sty}{\usepackage{upquote}}{}
\IfFileExists{microtype.sty}{% use microtype if available
  \usepackage[]{microtype}
  \UseMicrotypeSet[protrusion]{basicmath} % disable protrusion for tt fonts
}{}
\makeatletter
\@ifundefined{KOMAClassName}{% if non-KOMA class
  \IfFileExists{parskip.sty}{%
    \usepackage{parskip}
  }{% else
    \setlength{\parindent}{0pt}
    \setlength{\parskip}{6pt plus 2pt minus 1pt}}
}{% if KOMA class
  \KOMAoptions{parskip=half}}
\makeatother
\usepackage{xcolor}
\IfFileExists{xurl.sty}{\usepackage{xurl}}{} % add URL line breaks if available
\IfFileExists{bookmark.sty}{\usepackage{bookmark}}{\usepackage{hyperref}}
\hypersetup{
  pdftitle={s670-HW4},
  pdfauthor={Ruchik Rohit Dama},
  hidelinks,
  pdfcreator={LaTeX via pandoc}}
\urlstyle{same} % disable monospaced font for URLs
\usepackage[margin=1in]{geometry}
\usepackage{graphicx}
\makeatletter
\def\maxwidth{\ifdim\Gin@nat@width>\linewidth\linewidth\else\Gin@nat@width\fi}
\def\maxheight{\ifdim\Gin@nat@height>\textheight\textheight\else\Gin@nat@height\fi}
\makeatother
% Scale images if necessary, so that they will not overflow the page
% margins by default, and it is still possible to overwrite the defaults
% using explicit options in \includegraphics[width, height, ...]{}
\setkeys{Gin}{width=\maxwidth,height=\maxheight,keepaspectratio}
% Set default figure placement to htbp
\makeatletter
\def\fps@figure{htbp}
\makeatother
\setlength{\emergencystretch}{3em} % prevent overfull lines
\providecommand{\tightlist}{%
  \setlength{\itemsep}{0pt}\setlength{\parskip}{0pt}}
\setcounter{secnumdepth}{-\maxdimen} % remove section numbering
\ifluatex
  \usepackage{selnolig}  % disable illegal ligatures
\fi

\title{s670-HW4}
\author{Ruchik Rohit Dama}
\date{10/12/2021}

\begin{document}
\maketitle

\hypertarget{fit-a-model-to-predict-a-movies-imdb-rating-variableaveragerating-by-year-startyearand-length-runtimeminutes.-you-will-have-to-make-a-number-of-modeling-choices}{%
\subsubsection{1. Fit a model to predict a movie's IMDB rating
(variableaverageRating) by year (startYear)and length (runtimeMinutes.)
You will have to make a number of modeling
choices:}\label{fit-a-model-to-predict-a-movies-imdb-rating-variableaveragerating-by-year-startyearand-length-runtimeminutes.-you-will-have-to-make-a-number-of-modeling-choices}}

\hypertarget{a-do-you-need-any-transformations}{%
\paragraph{(a) Do you need any
transformations?}\label{a-do-you-need-any-transformations}}

\hypertarget{i-tried-multiple-transformations.-i-used-logarithmic-transformations-and-square-transformations-but-there-was-not-any-difference-in-the-results-so-i-did-not-perform-any-transformations-on-the-data.}{%
\paragraph{I tried multiple transformations. I used logarithmic
transformations, and square transformations, but there was not any
difference in the results so, I did not perform any transformations on
the
data.}\label{i-tried-multiple-transformations.-i-used-logarithmic-transformations-and-square-transformations-but-there-was-not-any-difference-in-the-results-so-i-did-not-perform-any-transformations-on-the-data.}}

\hypertarget{b-should-you-fit-a-linear-model-or-something-curved}{%
\paragraph{(b) Should you fit a linear model or something
curved?}\label{b-should-you-fit-a-linear-model-or-something-curved}}

\hypertarget{i-used-both-weighted-and-non-weightd-lm-and-gam-but-the-residual-plots-for-weighted-gam-was-gave-better-result-than-other-models.i-have-used-smoothing-transformation-on-runtimeminutes-as-it-gave-better-results.}{%
\paragraph{I used both weighted and non-weightd lm and gam, but the
residual plots for weighted gam was gave better result than other
models.I have used smoothing transformation on runtimeMinutes as it gave
better
results.}\label{i-used-both-weighted-and-non-weightd-lm-and-gam-but-the-residual-plots-for-weighted-gam-was-gave-better-result-than-other-models.i-have-used-smoothing-transformation-on-runtimeminutes-as-it-gave-better-results.}}

\hypertarget{c-is-an-additive-model-adequate}{%
\paragraph{(c) Is an additive model
adequate?}\label{c-is-an-additive-model-adequate}}

\hypertarget{the-gam-model-showed-better-results-than-lm-model.-but-gam-model-is-not-adequate.-it-has-r-square-value-of-0.214-and-deviance-explained-22.8.}{%
\paragraph{The gam model showed better results than lm model. but, gam
model is not adequate. It has R square value of 0.214 and deviance
explained
22.8\%.}\label{the-gam-model-showed-better-results-than-lm-model.-but-gam-model-is-not-adequate.-it-has-r-square-value-of-0.214-and-deviance-explained-22.8.}}

\hypertarget{d-do-you-need-to-filter-out-or-downweight-tail-values-to-prevent-the-fit-from-being-dominated-by-outliers}{%
\paragraph{(d) Do you need to filter out or downweight tail values to
prevent the fit from being dominated by
outliers?}\label{d-do-you-need-to-filter-out-or-downweight-tail-values-to-prevent-the-fit-from-being-dominated-by-outliers}}

\hypertarget{since-the-outliers-do-not-contribute-in-the-data-we-filter-out-the-tail-values-for-the-runtimeminutes.-usually-movies-are-in-the-range-of-50-minutes-to-200-minutes.-so-i-am-dropping-all-movies-with-length-lesser-than-50-and-greater-than-200-minutes.}{%
\paragraph{SInce, the outliers do not contribute in the data, we Filter
out the tail values for the runtimeMinutes. Usually, movies are in the
range of 50 minutes to 200 minutes. So, I am dropping all movies with
length lesser than 50 and greater than 200
minutes.}\label{since-the-outliers-do-not-contribute-in-the-data-we-filter-out-the-tail-values-for-the-runtimeminutes.-usually-movies-are-in-the-range-of-50-minutes-to-200-minutes.-so-i-am-dropping-all-movies-with-length-lesser-than-50-and-greater-than-200-minutes.}}

\hypertarget{e-should-you-weight-by-number-of-votes}{%
\paragraph{(e) Should you weight by number of
votes?}\label{e-should-you-weight-by-number-of-votes}}

\hypertarget{after-using-both-weighted-and-non-weightd-lm-and-gam-i-can-say-that-the-weighted-gam-model-showed-better-result-than-non-weighted-gam-model.-the-non-weighted-gam-model-had-a-r-square-rating-of-0.0268-and-deviance-explained-was-2.68-whereas-the-weighed-gam-model-had-r-square-value-of-0.214-and-deviance-explained-22.8.}{%
\paragraph{After using both weighted and non-weightd lm and gam, I can
say that the weighted gam model showed better result than non-weighted
gam model. the non-weighted gam model had a R-square rating of 0.0268
and deviance explained was 2.68\% whereas the weighed gam model had R
square value of 0.214 and deviance explained
22.8\%.}\label{after-using-both-weighted-and-non-weightd-lm-and-gam-i-can-say-that-the-weighted-gam-model-showed-better-result-than-non-weighted-gam-model.-the-non-weighted-gam-model-had-a-r-square-rating-of-0.0268-and-deviance-explained-was-2.68-whereas-the-weighed-gam-model-had-r-square-value-of-0.214-and-deviance-explained-22.8.}}

\hypertarget{draw-one-set-of-faceted-plots-to-display-the-model-either-condition-on-year-or-lengthwhichever-seems-to-you-to-be-more-interesting.-choose-a-sensible-number-of-panels.-briefly-describe-what-this-set-of-plots-shows-you.}{%
\subsubsection{2. Draw ONE set of faceted plots to display the model ---
either condition on year or length,whichever seems to you to be more
interesting. Choose a sensible number of panels. Briefly describe what
this set of plots shows
you.}\label{draw-one-set-of-faceted-plots-to-display-the-model-either-condition-on-year-or-lengthwhichever-seems-to-you-to-be-more-interesting.-choose-a-sensible-number-of-panels.-briefly-describe-what-this-set-of-plots-shows-you.}}

\hypertarget{using-10-as-the-panel-number-i-am-drawing-faceted-plots-for-the-weighted-gam-model-with-the-condtion-in-runtimeminutes-and-faceting-it-on-the-year-column.-we-can-see-that-the-average-ratings-of-the-movies-increase-as-the-runtime-increases-but-the-pattern-in-the-graph-is-similar-for-every-decade-so-from-the-faceted-plots-we-can-say-that-though-the-raitngs-of-the-movies-increase-with-increase-in-runtime-but-it-does-not-show-any-dependency-on-startyear.}{%
\paragraph{Using 10 as the panel number, I am drawing faceted plots for
the weighted gam model with the condtion in runtimeMinutes and faceting
it on the year column. We can see that the average ratings of the movies
increase as the runtime increases, but the pattern in the graph is
similar for every decade, so from the faceted plots, we can say that
though the raitngs of the movies increase with increase in runtime but
it does not show any dependency on
startYear.}\label{using-10-as-the-panel-number-i-am-drawing-faceted-plots-for-the-weighted-gam-model-with-the-condtion-in-runtimeminutes-and-faceting-it-on-the-year-column.-we-can-see-that-the-average-ratings-of-the-movies-increase-as-the-runtime-increases-but-the-pattern-in-the-graph-is-similar-for-every-decade-so-from-the-faceted-plots-we-can-say-that-though-the-raitngs-of-the-movies-increase-with-increase-in-runtime-but-it-does-not-show-any-dependency-on-startyear.}}

\hypertarget{draw-a-raster-and-contour-plot-or-other-3d-plot-of-your-choice-to-further-display-yourmodel.-the-plot-should-show-predictions-for-the-majority-of-movie-years-and-lengths-youdont-have-to-show-outliers.-briefly-describe-what-if-anything-this-plot-shows-you-that-yourplot-for-question-2-didnt.}{%
\subsubsection{3. Draw a raster-and-contour plot (or other ``3D'' plot
of your choice) to further display yourmodel. The plot should show
predictions for the majority of movie years and lengths (youdon't have
to show outliers.) Briefly describe what, if anything, this plot shows
you that yourplot for question 2
didn't.}\label{draw-a-raster-and-contour-plot-or-other-3d-plot-of-your-choice-to-further-display-yourmodel.-the-plot-should-show-predictions-for-the-majority-of-movie-years-and-lengths-youdont-have-to-show-outliers.-briefly-describe-what-if-anything-this-plot-shows-you-that-yourplot-for-question-2-didnt.}}

\hypertarget{from-the-contour-plots-we-cans-see-that-ratings-increase-as-runtime-increases.-for-example-when-the-runtime-is-near-50-the-ratings-are-near-6-or-7-and-as-runtime-reaches-100-we-get-some-movies-with-ratings-near-8-and-as-the-runtime-increases-to-150-200-we-get-moves-with-ratings-9-and-above.-we-can-also-see-the-there-dependency-on-the-year-column.-we-can-see-that-the-as-the-year-increases-the-ratings-also-change.the-shorter-movies-the-older-movies-have-slightly-higher-rating-than-the-recent-movies.-even-for-the-longer-movies-the-older-movies-have-higher-rating-than-recent-movies.-for-example-the-shorter-movies-of-50-mintues-the-older-movies-have-ratings-of-7-whereas-the-recent-movies-have-rating-of-6.-similarly-for-longer-movies-of-150-200-minutes-the-older-movies-have-higher-ratings-of-9-and-above-but-the-recent-movies-have-there-ratings-till-8.-in-the-facet-plots-we-couldnt-see-how-does-the-year-affects-the-ratings-of-the-movies-but-in-the-contour-plots-we-can-see-the-affect-of-year-on-the-ratings}{%
\paragraph{From the contour plots, we cans see that ratings increase as
runtime increases. For Example, when the runtime is near 50, the ratings
are near 6 or 7 and as runtime reaches 100, we get some movies with
ratings near 8 and as the runtime increases to 150-200 we get moves with
ratings 9 and above. We can also see the there dependency on the year
column. We can see that the as the year increases ,the ratings also
change.The shorter movies , the older movies have slightly higher rating
than the recent movies. Even for the longer movies, the older movies
have higher rating than recent movies. For Example, the shorter movies
of 50 mintues, the older movies have ratings of 7 whereas the recent
movies have rating of 6. SImilarly, for longer movies of 150-200
minutes, the older movies have higher ratings of 9 and above but the
recent movies have there ratings till 8. In the facet plots we couldn't
see how does the year affects the ratings of the movies, but in the
contour plots we can see the affect of year on the
ratings}\label{from-the-contour-plots-we-cans-see-that-ratings-increase-as-runtime-increases.-for-example-when-the-runtime-is-near-50-the-ratings-are-near-6-or-7-and-as-runtime-reaches-100-we-get-some-movies-with-ratings-near-8-and-as-the-runtime-increases-to-150-200-we-get-moves-with-ratings-9-and-above.-we-can-also-see-the-there-dependency-on-the-year-column.-we-can-see-that-the-as-the-year-increases-the-ratings-also-change.the-shorter-movies-the-older-movies-have-slightly-higher-rating-than-the-recent-movies.-even-for-the-longer-movies-the-older-movies-have-higher-rating-than-recent-movies.-for-example-the-shorter-movies-of-50-mintues-the-older-movies-have-ratings-of-7-whereas-the-recent-movies-have-rating-of-6.-similarly-for-longer-movies-of-150-200-minutes-the-older-movies-have-higher-ratings-of-9-and-above-but-the-recent-movies-have-there-ratings-till-8.-in-the-facet-plots-we-couldnt-see-how-does-the-year-affects-the-ratings-of-the-movies-but-in-the-contour-plots-we-can-see-the-affect-of-year-on-the-ratings}}

\hypertarget{answer-the-substantive-question-do-longer-movies-tend-to-get-higher-imdb-ratings-after-accounting-for-their-year-of-release-the-answer-will-likely-be-more-complicated-than-yesor-no.}{%
\subsubsection{4. Answer the substantive question: Do longer movies tend
to get higher IMDB ratings, after accounting for their year of release?
(The answer will likely be more complicated than ``yes''or
``no.'')}\label{answer-the-substantive-question-do-longer-movies-tend-to-get-higher-imdb-ratings-after-accounting-for-their-year-of-release-the-answer-will-likely-be-more-complicated-than-yesor-no.}}

\hypertarget{from-the-above-contour-plot-we-can-say-that-the-longer-movies-tend-to-get-higher-imdb-ratings-and-it-also-shows-some-dependency-on-year-of-release.-as-seen-in-contour-plots-the-range-for-the-ratings-of-movies-tend-to-change-as-the-year-increases.-this-means-for-older-movies-we-get-a-much-broader-range-for-example-for-the-year-1920-the-shorter-movies-have-there-ratings-starting-from-6-whereas-the-longer-movies-have-there-ratings-staring-from-9-so-the-range-is-starts-from-6-and-it-crosses-9-but-for-recent-movies-the-range-is-much-smaller-for-example-in-the-year-2000-the-shorter-movies-has-the-rating-of-6-but-the-longer-movies-have-the-rating-of-8-so-there-range-starts-from-6-and-reaches-8.-so-the-longer-movies-tend-to-get-higher-imdb-ratings-even-after-accounting-their-year-of-releasr-but-the-range-for-the-rating-differs-and-it-gets-smaller-as-the-year-progresses.}{%
\paragraph{From the above contour plot we can say that, the longer
movies tend to get higher IMDB ratings and it also shows some dependency
on year of release. As seen in contour plots, the range for the ratings
of movies tend to change as the year increases. This means, for older
movies we get a much broader range, For example, for the year 1920, the
shorter movies have there ratings starting from 6 whereas the longer
movies, have there ratings staring from 9, so the range is starts from 6
and it crosses 9, but for recent movies, the range is much smaller, for
example in the year 2000, the shorter movies has the rating of 6 but the
longer movies have the rating of 8, so there range starts from 6 and
reaches 8. So, the longer movies tend to get higher IMDB ratings, even
after accounting their year of releasr but the range for the rating
differs and it gets smaller as the year
progresses.}\label{from-the-above-contour-plot-we-can-say-that-the-longer-movies-tend-to-get-higher-imdb-ratings-and-it-also-shows-some-dependency-on-year-of-release.-as-seen-in-contour-plots-the-range-for-the-ratings-of-movies-tend-to-change-as-the-year-increases.-this-means-for-older-movies-we-get-a-much-broader-range-for-example-for-the-year-1920-the-shorter-movies-have-there-ratings-starting-from-6-whereas-the-longer-movies-have-there-ratings-staring-from-9-so-the-range-is-starts-from-6-and-it-crosses-9-but-for-recent-movies-the-range-is-much-smaller-for-example-in-the-year-2000-the-shorter-movies-has-the-rating-of-6-but-the-longer-movies-have-the-rating-of-8-so-there-range-starts-from-6-and-reaches-8.-so-the-longer-movies-tend-to-get-higher-imdb-ratings-even-after-accounting-their-year-of-releasr-but-the-range-for-the-rating-differs-and-it-gets-smaller-as-the-year-progresses.}}

\end{document}
